Este es el texto del primer informe automático generado por una acción de GitHub Actions.\\

Ahora vamos a probar el código generado para la ejecución automática del primer informe.

\begin{table}[H]
\centering
\begin{tabular}{|c|c|}\hline
\textbf{Provincia} & \textbf{Capital} \\ \hline
Pichincha & Quito \\ \hline
Guayas & Guayaquil \\ \hline
Azuay & Cuenca \\ \hline
\end{tabular}
\caption{Provincias y sus capitales}
\end{table}

Otro párrafo de prueba.
\begin{knitrout}
\definecolor{shadecolor}{rgb}{0.969, 0.969, 0.969}\color{fgcolor}\begin{kframe}
\begin{alltt}
\hlnum{3}\hlopt{+}\hlnum{5}\hlopt{-}\hlnum{2}
\end{alltt}
\begin{verbatim}
## [1] 6
\end{verbatim}
\begin{alltt}
\hlkwd{rnorm}\hldef{(}\hlnum{5}\hldef{,}\hlkwc{mean} \hldef{=} \hlnum{12}\hldef{,} \hlkwc{sd} \hldef{=} \hlnum{4} \hldef{)}
\end{alltt}
\begin{verbatim}
## [1]  6.143355 16.245674 14.945751 17.320050  7.910795
\end{verbatim}
\end{kframe}
\end{knitrout}


A continuación nuestro segundo chunk
\begin{knitrout}
\definecolor{shadecolor}{rgb}{0.969, 0.969, 0.969}\color{fgcolor}
\includegraphics[width=\maxwidth]{figure/chunk02-1} 
\end{knitrout}
